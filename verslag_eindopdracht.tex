\documentclass[a4paper]{article}

\usepackage{xcolor}
\usepackage{fancyheadings}
\usepackage{booktabs}
\usepackage{enumitem}

\newcommand{\todo}[1]{\textcolor{red}{[#1]}}
\newcommand{\1}[0]{\'{e}\'{e}n}
\newcommand{\code}[1]{\texttt{#1}}

\renewcommand\labelitemi{-}
\renewcommand{\figurename}{Figuur}
\renewcommand{\tablename}{Tabel}

\lhead{Open Universiteit}
\chead{IM0102, Design patterns}
\rhead{Eindopdracht}

\begin{document}
\pagestyle{fancy}

\section*{Studentgegevens}
\begin{description}
	\item [Cursuscode] IM0102
	\item JabberPoint
    \item Jan Jaap Sandee --- 852025385
	\item Gerralt Gottemaker --- 852083852
\end{description}

\section*{Aanpak}
Om de opdracht succesvol af te ronden, is het zaak om goed samen te werken. We hebben het geluk collega's te zijn, wat het communiceren een aanzienlijk stuk gemakkelijker maakt. In dit hoofdstuk staat de aanpak beschreven die we gekozen hebben. De volgorde waarin het opgeschreven is, is ook de volgorde waarin we gewerkt hebben. Verder staat per item beschreven wie hieraan gewerkt heeft. Als niet vermeld wordt wie eraan gewerkt heeft, hebben beide teamgenoten eraan gewerkt.

Er is begonnen met het opzetten van een Git repository waarin Jan Jaap Gerralt en beide examinatoren heeft toegevoegd. Hier is vervolgens de beginsituatie van JabberPoint toegevoegd. Even later is hier ook een begin van het verslag en een TODO lijst aan toegevoegd. Het verslag is toegevoegd omdat het een LaTeX bestand is, welke prima door versiebeheer beheerd kan worden. Verder heeft Gerralt een Google Drive folder gemaakt en Jan Jaap uitgenodigd. Hier zullen diagrammen (met Draw.io) en andere belangrijke bestanden die niet geschikt zijn voor Git versiebeheer worden bijgehouden. Als laatst heeft Gerralt een Project Board aangemaakt waar overzichtelijk de TODO's wat betreft code en documentatie bijgehouden kunnen worden.

Hierna heeft Jan Jaap een eenvoudig diagram gemaakt van de huidige situatie van JabberPoint. Dit geeft een duidelijker beeld van de pijnpunten en maakt de geplande refactoring overzichtelijker.

Vervolgens is er een Probleem Analyse gemaakt waarbij alleen is gekeken naar de functionaliteit van het systeem. De zogenaamde Mankala aanpak staat hierbij centraal. Gedurende het maken van de analyse is nog niet gekeken naar bestaande code van JabberPoint, om zo vooroordelen uit te sluiten. Hier is bewust voor gekozen omdat we het idee hebben dat het zo gemakkelijker is om te denken in termen van het domein en niet direct te denken in klassenamen. Jan Jaap heeft hiervan het opzetje gemaakt, welke vervolgens uitgebreid is door Gerralt. Hierna is samen gewerkt aan de uiteindelijke versie van de probleemanalyse. Beiden kwamen tot vergelijkbare conclusies wat betreft de Probleem Analyse.

Aan de hand van de probleemanalyse is een klassendiagram opgebouwd. Hierbij zijn de principes die genoemd worden in het tekstboek gevolgd, waarbij erg gelet is op het principe dat patterns de context creëren van onderliggende patterns. De design principes zoals high cohesion en low coupling staan hierbij natuurlijk ook centraal. In het verslag zijn vervolgens de verschillende (belangrijke) classes en structuren toegelicht.

Het klassendiagram is samen gemaakt, in persoon. Er was een whiteboard en een (gedeeld) beeldscherm beschikbaar, waar we onze idee\"{e}n op kwijt konden. Het ontwerp kreeg steeds meer vorm door een top-down aanpak, waarbij we steeds de context cre\"{e}erden voor het volgende pattern. De scheiding GUI / functionaliteit was het eerst aan de beurt, om vervolgens tot een steeds gedetailleerder ontwerp te komen

Nadat het ontwerp afgerond was, is een nieuw project gemaakt. Gerralt heeft de boilerplate code gemaakt en heeft de eerste functionaliteit gebouwd. Vervolgens is voor wat nog open stond issues gemaakt op het Project Board op Github. Dit is vervolgens door zowel Gerralt als Jan Jaap opgepakt.

Als laatst is de documentatie geschreven door beide teamleden. Jan Jaap heeft hier een eerste versie voor gemaakt, die uitgebreid is door Gerralt. Vervolgens hebben beide teamleden de documentatie geschreven. Dit alles is op het laatst gecontroleerd op taal- en spelling.

\section{Probleemanalyse}
Om JabberPoint goed te kunnen refactoren, is een probleemanalyse van de casus gemaakt. Deze probleemanalyse is op basis van de Mankala aanpak gedaan. Deze aanpak bestaat uit een aantal stappen, welke een steeds duidelijker beeld van het domein moeten opleveren. In de volgende koppen staan deze stappen uitgewerkt.

\subsection{Dingen}
We beginnen met het vastleggen van de entiteiten van het domein. Deze entiteiten, of "dingen", worden opgesomd in Tabel \ref{table:dingen}. De entiteiten die daar genoemd worden, zijn de entiteiten die van belang zijn voor het domein. Entiteiten die te maken hebben met de gebruikersinterface worden hier niet genoemd.
\begin{table}[!h]
\centering
	\begin{tabular}{lcl}
	\toprule
 	Concept & Sub-concept & Uitleg \\ \midrule
 	JabberPoint & - & Het programma. Kan presentaties afspelen.\\
 	Presentatie & - & Een enkele presentatie. Bevat slides.\\
 	Slide & - & Een pagina. Bevat elementen en een titel.\\
 	Element & Tekst & Bevat tekst voor op een slide.\\
 	 & Afbeelding & Bevat een afbeelding voor een slide.\\
 	Level & - & Het level van een element bepaalt wie de ouder is.\\
 	Stijl & - & Bevat info over het uiterlijk van een element.\\
 	Bestand & XML & Formaat waarin presentaties opgeslagen worden.\\
 	\bottomrule
	\end{tabular}
\caption{Lijst van entiteiten (dingen) binnen het domein.}
\label{table:dingen}
\end{table}
\\
Op het moment zijn er twee typen elementen: elementen die tekst bevatten en elementen die een afbeelding bevatten. Om het systeem zo flexibel mogelijk te maken, zal in het ontwerp rekening gehouden moeten worden met elementen met andere typen. Zo is het bijvoorbeeld denkbaar dat er in de toekomst een video element toegevoegd moet worden of iets dergelijks.
\\\\
Verder is het belangrijk om andere bestandsformaten alvast te ondersteunen in het ontwerp. Hierdoor behoudt het systeem zijn flexibiliteit. Op het moment wordt alleen XML ondersteund voor zowel het wegschrijven als het lezen. Echter zal in het ontwerp dus rekening gehouden worden met later toe te voegen bestandsformaten zoals JSON. Deze bestandsformaten zullen niet toegevoegd worden in zowel ontwerp als implementatie, echter zullen we bij de keuzen toelichten hoe we wel rekening houden met deze formaten.

\subsection{Acties}
Om de acties binnen het domein te kunnen gebruiken voor de analyse en het klassendiagram, is bepaalde informatie nodig. Om deze informatie te vergaren, wordt per actie de volgende vragen beantwoord:
\begin{itemize}[noitemsep]
\item Wie of wat heeft het initiatief om de actie te starten?
\item Aan welke regels moet voldaan worden voordat de actie uitgervoerd kan worden?
\item Wie of wat is gerelateerd aan de actie?
\item Is er aanvullende informatie nodig om de actie te kunnen uitvoeren?
\end{itemize}

In principe geldt voor elke actie dat het initiatief in de handen van de gebruiker ligt. Deze zal elke actie starten middels de grafische user interface (GUI). Echter zullen we bij het beantwoorden van de vraag ervan uitgaan dat de gebruiker de betreffende actie al begonnen is. Het initiatief ligt daarom bij de klasse / subroutine binnen het softwaresysteem en deze wordt dan ook genoemd. Om dezelfde reden wordt ook de GUI niet bij elk punt genoemd. Het moge echter duidelijk zijn dat de GUI voor de aansturing van het functionele deel van JabberPoint zorgt.

Deze sectie is vrij uitgebreid met als simpele reden dat deze sectie in principe de gehele functionele werking van JabberPoint toelicht. Hierdoor hoeft er in hoofdstuk \ref{sec:source} alleen aandacht besteed te worden aan de technische werking van het programma.

\subsubsection{Programma starten}
Wanneer het programma wordt gestart, zal JabberPoint opgestart worden met bepaalde standaard waarden. Bovenop JabberPoint zal een grafische user interface (GUI) gebouwd worden. Hoe de GUI communiceert met JabberPoint, is te vinden in hoofdstuk \ref{sec:ontwerp}. Om deze actie uit te voeren, is het aan te raden om een bestandsnaam mee te geven in de parameters. Wanneer dit niet gebeurt, zal het systeem opstarten met een standaard presentatie. Als aanvullende informatie is het goed om te weten dat de GUI hier los gezien moet worden van JabberPoint. JabberPoint is dus het functionele deel van het programma.

\subsubsection{Presentatie openen}
\label{subsub:presOpenen}
De tweede actie is het openen van een presentatie. De actie wordt gestart door het afhandelen van een gebeurtenis uit het menu en zal door moeten druppelen naar het functionele deel van JabberPoint. De actie wordt gestart door het afhandelen van een gebeurtenis uit het menu en zal door moeten druppelen naar het functionele deel van JabberPoint. JabberPoint en de Presentatie zijn gerelateerd aan deze actie. Verder moet genoemd worden dat momenteel geen bestand van het systeem gekozen kan worden. JabberPoint zal dus een standaard bestand (test.xml) openen. In het ontwerp zal wel rekening gehouden worden met een mogelijke uitbreiding om meerdere bestandsformaten te ondersteunen.

\subsubsection{Nieuwe presentatie starten}
De volgende actie is het starten van een nieuwe presentatie. Deze presentatie is volledig leeg. Wat het starten van de actie en de gerelateerde entiteiten betreft, geldt hetzelfde als bij de actie Presentatie openen. Het is verder goed om te noemen dat het starten van een nieuwe presentatie relatief weinig nut heeft. Er is immers geen functionaliteit om de presentatie aan te passen via de GUI.

\subsubsection{Presentatie opslaan}
Verder kunnen presentaties worden opgeslagen. Het gaat dan om de presentatie die op dat moment getoond wordt door de GUI. Deze actie wordt wederom gestart door het menu en ook hier geldt dat de actie door moet druppelen naar het functionele deel. Het opslaan lukt alleen als er een presentatie geopend is. Gelukkig is er altijd een presentatie geopend, dus dit mag geen problemen opleveren. Ook hier geldt dat JabberPoint, de Presentatie en het bestandsformaat gerelateerd zijn. Verder moet worden opgemerkt dat er geen feedback gegeven wordt wanneer het opslaan lukt. Dit is echter puur functioneel. Het zal dus niet in het ontwerp terug komen, maar wel in de implementatie. Ook hier wordt in het ontwerp rekening gehouden met eventuele later toe te voegen bestandsformaten anders dan XML.

\subsubsection{Slide weergeven}
\label{subsub:slideTonen}
Wanneer het programma wordt gestart, wordt een presentatie geopend (zie kopje \ref{subsub:presOpenen}). Wanneer dit gebeurt, wordt de eerste slide van deze presentatie weergegeven op het scherm. In eerste instantie wordt alleen de titel van deze slide weergegeven. Hoe de overige items op het scherm moeten komen, wordt geschreven in een aantal van de volgende kopjes. Deze actie is dus gerelateerd van de actie welke het openen van een presentatie beschrijft. Ook JabberPoint en de Presentatie zijn gerelateerd. Verder gelden er enkele regels waar eerst aan voldaan moet worden voordat deze actie uitgevoerd kan worden. Deze regels worden in paragraaf \ref{sub:regels} genoemd.

\subsubsection{Volgende item van slide weergeven}
\label{subsub:volgendeItem}
Dit is een van de acties die voortkomt uit de Feature request. Wanneer er op het pijltje naar beneden gedrukt wordt, zal het volgende item worden weergegeven. Hoe het item wordt weergegeven, hangt volledig af van de Stijl van een item. Alleen level 1 items zullen deze actie ondersteunen. Alle onderliggende items van level 1 items worden tegelijkertijd getekend op het moment dat deze actie uitgevoerd wordt. Wanneer deze actie vervolgens nog een keer uitgevoerd wordt, wordt het volgende level 1 item plus onderliggende items getekend. Gerelateerd aan deze actie zijn de entiteiten JabberPoint, de Presentatie, Slide, item (Element) en Stijl. Verder is de actie uit kopje \ref{subsub:volgendeSlide} gerelateerd.

\subsubsection{Laatste item van slide verbergen}
\label{subsub:vorigeItem}
Ook deze actie komt voort uit de Feature request en hangt nauw samen met de vorige actie. Wanneer er op het pijltje naar boven wordt gedrukt, wordt het laatst verschenen level 1 item weer verborgen. Ook hier geldt dat alle onderliggende items (de kinderen van het betreffende level 1 item) worden verborgen. Gerelateerd aan deze actie zijn JabberPoint, de Presentatie, Slide en item (Element). Ook hier geldt dat er een andere actie gerelateerd is, namelijk de actie uit kopje \ref{subsub:vorigeSlide}.

\subsubsection{Alle items van slide in \1 keer tonen}
\label{subsub:alleItems}
Het is ook mogelijk om alle items van een bepaalde slide in \1 keer te tonen. Wederom gaat het hier om een actie die voortkomt uit de Feature request. Wanneer op het pijltje naar rechts gedrukt wordt, worden alle nog verborgen items van een slide allemaal weergegeven. In principe is dit niets anders dan een aantal keer de actie van kopje \ref{subsub:volgendeItem} aanroepen. Daarom geldt dat alle gerelateerde regels en entiteiten hetzelfde zijn.

\subsubsection{Alle items van slide in \1 keer verbergen}
De laatste actie die uit de Feature request komt is het in \1 keer verbergen van alle op dat moment zichtbare items. Deze actie wordt uitgevoerd wanneer er op het pijltje naar links gedrukt wordt. Net als bij de vorige actie, geldt hier dat deze actie niets anders is dan een aantal keer een bepaalde actie uitvoeren, namelijk de actie van kopje \ref{subsub:vorigeItem}. Ook hier geldt daarom dat alle gerelateerde regels en entiteiten hetzelfde zijn.

\subsubsection{Volgende slide}
\label{subsub:volgendeSlide}
Naar de volgende slide gaan is de volgende actie. Deze actie zal gestart worden wanneer het volgende item van de slide weergegeven moet worden, maar deze niet meer beschikbaar is. Wanneer dit gebeurt, zal er gekeken worden of er nog een slide beschikbaar is. Mocht dit het geval zijn, zal deze worden getoond zoals beschreven in het kopje \ref{subsub:slideTonen}. Vervolgens kan de actie uit het kopje \ref{subsub:volgendeItem} gebruikt worden om de items te tonen of de actie uit kopje \ref{subsub:alleItems} om alle items in \1 keer te tonen. Deze actie moet dus, voordat het uitgevoerd kan worden, aan bovenstaande regels voldoen. De bovengenoemde acties zijn dus gerelateerd aan deze actie. Ook de GUI en de entiteiten JabberPoint en de Presentatie zijn gerelateerd.

\subsubsection{Vorige slide}
\label{subsub:vorigeSlide}
Naast naar de volgende slide gaan is de tegenovergestelde actie ook mogelijk. Wanneer de actie uit kopje \ref{subsub:vorigeItem} (vorige item) uitgevoerd wordt zonder dat er nog items op het scherm zichtbaar zijn, wordt er naar de vorige slide genavigeerd. Hierbij moet aan de regel voldaan worden dat er een vorige slide beschikbaar is. Mocht er niet aan deze regel voldaan worden, zal de actie simpelweg niet uitgevoerd worden. De genoemde regel is gerelateerd aan deze actie, evenals de genoemde actie. Verder zijn de entiteiten JabberPoint, Presentatie en Slide gerelateerd.

\subsubsection{Naar specifieke slide}
Ook is het mogelijk om rechtstreeks naar een specifieke slide te navigeren. De GUI zal vragen om een getal, welke vervolgens gecontroleerd wordt. Als blijkt dat de gebruiker een ongeldig getal heeft ingevuld (ongeldig betekent lager dan 1 of hoger dan het aantal slides), wordt de actie niet uitgevoerd. Mocht het een geldig getal betreffen, zal het systeem de juiste slide tonen zoals beschreven in het kopje \ref{subsub:slideTonen}. Deze actie is dan ook gerelateerd. Verder zijn de entiteiten JabberPoint, Presentatie en Slide gerelateerd, evenals de bovengenoemde regel.

\subsubsection{Informatie "Over / about" weergeven}
Als laatst is het mogelijk om bepaalde auteurs- en versieinformatie te tonen. Dit zal gedaan worden via het menu van de GUI. De GUI zal vragen aan JabberPoint wat de te tonen informatie is en deze tonen. Daarom is alleen de entiteit JabberPoint gerelateerd. Er is verder geen informatie nodig om deze actie uit te voeren. 

\subsection{Regels}
\label{sub:regels}

Er zijn regels waar bepaalde acties aan moeten voldoen voordat de betreffende actie uitgevoerd mag worden. Ook zijn er regels waar het hele systeem aan moet voldoen om succesvol uitgevoerd te kunnen worden. Verder zijn er voor deze opdracht ook relevante regels waar de gebruikersinterface mee te maken krijgt. Al deze regels zijn opgesomd in Tabel \ref{table:regels}. Elke regel bevat ook een korte beschrijving.

\begin{table}[!h]
\centering
	\begin{tabular}{ll}
	\toprule
 	Regel & Omschrijving \\ \midrule
 	Presentatie openen & Wanneer de applicatie gestart wordt met bestandsnaam, \\& wordt deze presentatie geopend. \\
 	 Bestandsformaat & In de huidige vorm van JabberPoint, wordt alleen XML \\& Ondersteund als formaat. Mogelijke Feature request.\\
 	Demo presentatie & Wanneer de applicatie gestart wordt zonder bestandsnaam, \\& wordt er een demo presentatie geopend.\\
 	Huidige presentatie & Er is ten alle tijden precies \1 presentatie open.\\
 	Minstens \1 slide & Om een presentatie weer te kunnen geven, moet er ten \\& minste \1 slide aanwezig zijn in de presentatie.\\
 	Level van items & Level 1 items zijn de root items. Hogere levels zijn \\& onderliggende items van level 1 items.\\
 	Styling van items & De styling van items moet per item kunnen verschillen.\\
 	Soort styling & Onder styling valt: Kleur, font, lettergrootte, indentatie \\& en lead. Lead is de afstand tot de voorgaande regel.\\
 	Vaste styling & Er is een voorgedefinieerde set styling beschikbaar.\\
 	Type van items & Items moeten van type kunnen verschillen. Het hangt \\& per type af of er wat gedaan wordt met de styling.\\
 	Beginstaat slide & Wanneer een slide weergegeven wordt, is initieel alleen \\& de titel zichtbaar. \\
 	Volgende slide & De volgende slide moet worden getoond als alle items \\& zichtbaar zijn en er een volgende slide beschikbaar is.\\
 	Vorige slide & De vorige slide moet worden getoond wanneer alleen de \\& titel zichtbaar is en er een vorige slide beschikbaar is.\\
 	Direct navigeren & Het moet mogelijk zijn om ook direct naar een volgende \\& of vorige slide te navigeren, zonder alle items te zien.\\
 	Specifieke slide & Er kan alleen naar een specifieke slide genavigeerd worden \\& wanneer er een geldige slidenummer is opgegeven.\\
 	Level bepalend & Het level van een element bepaalt wie de ouder van dat \\& element is. Hoe hoger het level, hoe dieper de structuur.\\
 	Edit niet mogelijk & In de huidige vorm van JabberPoint, is het editen van \\& slides niet mogelijk. Mogelijke Feature request.\\
 	Alles tonen & Het is mogelijk om in de inputbestanden aan te geven \\& of een slide alle items in \1 moet tonen of niet.\\
 	Standaard gedrag & Wanneer er geen waarde wordt aangegeven in de input, \\& wordt als standaard gedrag alleen de titel getoond.\\
  	\bottomrule
	\end{tabular}
\caption{Lijst van regels binnen het domein.}
\label{table:regels}
\end{table}

\section{Ontwerp}
\label{sec:ontwerp}
Het ontwerp is op te delen in vijf onderdelen die elk een eigen
verantwoordelijkheid hebben. JabberPoint de main class instantieert alleen de
SlideViewerFrame, welke vervolgens de rest van de GUI op zal bouwen.

\subsection{GUI}
De GUI is alleen verantwoordelijk voor de weergave aan de gebruiker.
SlideViewerFrame bouwt de gehele interface op. De GUI heeft geen idee van de
huidige presentaties, maar vertrouwd op een observer om te weten wanneer er iets
nieuws getoond moet worden.

\subsection{Controllers}
De belangrijkste controller is de JabberPointFacade, welke aanroepen ontvangt vanuit de View (GUI) en deze doorspeelt naar het Model (voornamelijk Presentatie). Er zit een PresentatieController tussen om te zorgen dat het Observer / Observable pattern goed toegepast kan worden. Zie hoofdstuk \ref{sec:keuzen} voor meer informatie hierover. Op de JabberPointFacade is het Singleton pattern toegepast om te zorgen dat er ten alle tijde maar \1 instantie van hoeft te zijn. Dit is daardoor ook de zogenaamde "single source of truth".
\\\\
De Facade registreert de Observer die is JabberPointComponent bij de Observable, de Presentatie. Dus de koppeling tussen Presentation en JabberPointComponent bestaat alleen maar via de observer pattern. Omdat zowel Presentation als JabberPointComponent gebruikmaken van respectievelijk een abstract class en een interface, weten Presentation en JabberPointComponent niet direct af van elkaars bestaan. Dit waarborgt het low coupling principe.

\subsection{Models}
Alle classes in het Model onderdeel zijn bedoeld om data op te slaan, aan te passen en inzichtelijk te maken. Hierbij is de hi\"{e}rarchie Presentation - Slide - SlideItem de belangrijkste. Via de Presentation is alle informatie over de huidige presentatie inzichtelijk. Ook bevatten deze classes methoden om getekend te kunnen worden. Deze methoden worden via de Facade aangeroepen en vervolgens naar de juiste plek doorgespeeld. Er zijn twee typen SlideItem: Een TextItem en een ImageItem. In het ontwerp is rekening gehouden met eventueel nieuwe typen, zoals een VideoItem of iets dergelijks. Een bijzondere aanpassing vanuit het oude systeem in het nieuwe systeem is het feit dat SlideItem nu een lijst heeft met SlideItems. Dit is bedoeld om de structuur van levels beter op te slaan. Door deze structuur is het in de toekomst ook makkelijker om andere datatypen, zoals JSON, te ondersteunen. JSON heeft namelijk de eigenschap dat het veel geneste structuren bevat. Deze geneste structuren zijn dankzij de nieuwe SlideItem structuur gemakkelijker te realiseren in een later stadium.
\\\\
Verder is het hier goed om te vermelden dat er precies vijf typen Style beschikbaar zijn. Hoe deze Style objecten ge\"{i}nstantieerd worden, wordt besproken in kopje \ref{sub:factories}.

\subsection{Factories}
\label{sub:factories}
De Factories zijn verantwoordelijk voor het aanmaken van alle objecten. De
FileParserFactory en FileEncoderFactory zijn er zodat de juiste strategy wordt
toegepast op een file, aangezien het Strategy Pattern geen rekening houdt met
objectcreatie.
\\\\
De Factories voor de Presentaties zijn er om creatie te scheiden. Er was een
argument geweest voor Presentaties die zijn eigen slides aanmaakt, en slides
die zijn eigen Items aanmaakt, echter dit zou betekenen dat de Presentatie
rechtstreeks toegang moet hebben tot de filedata wat niet wenselijk is.
\\\\
Verder is er een ObjectPool voor de Styles. Dit pattern wordt gebruikt om te
waarborgen dat er ten aller tijden precies vijf stijlen beschikbaar zijn. Hiermee wordt voorkomen dat er voor elke nieuwe SlideItem een nieuw Style object wordt aangemaakt. Dit is in kleine aantallen geen probleem, alleen kan het een probleem worden wanneer er Presentaties met bijvoorbeeld 1000 SlideItems geladen wordt. Om daar dan 1000 objecten met hooguit vijf variaties voor te maken, is een domme keuze in onze ogen. Een bijkomend voordeel is dat het op deze manier erg makkelijk wordt om nieuwe Styles toe te voegen aan de standaard pool. Mochten er ooit level 5 items ondersteund moeten worden, hoeft er slechts een Style aan de Pool toegevoegd te worden.

\subsection{FileHandlers}
De Filehandler is op zijn beurt weer een soort Facade voor de JabberFacade.
Namelijk JabberFacade heeft geen weet van hoe presentaties worden uitgelezen of
opgeslagen. Het enige wat JabberPointFacade doet, is de FileHandler opdracht geven om op te slaan / in te lezen. Vervolgens kiest de FileHandler de juiste handeling en strategie. De FileHandler regelt het inlezen en schrijven zelf niet: dit wordt namelijk gedelegeerd naar twee classes. Hierdoor kunnen aparte parsers en encoders bestaan. Ook zijn de taken lezen en schrijven te asymmetrisch om opgelost te worden door dezelfde class. Door de splitsing wordt het cohesion principe ook gewaarborgd.

\section{Keuzen}
\label{sec:keuzen}
Gedurende het ontwerpen, herontwerpen en implementeren, zijn er verschillende keuzen gemaakt. De belangrijkste keuzen zijn hier opgeschreven. De keuzen zijn redelijk technisch beschreven, om dichtbij het ontwerp en de implementatie te blijven.

\subsection{Facade}
Toen we het class diagram ontworpen hebben op basis van de analyse uit dit document, hebben we als uitgangspunt het Facade pattern gebruikt. Het was voor ons namelijk erg belangrijk om de GUI zoveel mogelijk te scheiden van de daadwerkelijke functionaliteit. Het Facade pattern stelde ons in staat om de context van de GUI direct goed te scheiden van de context van de functionaliteit. Vervolgens zijn we met deze twee contexten verder gaan werken met andere design patterns.

\subsection{Observer}
Zoals hierboven beschreven, is de GUI gescheiden van de functionaliteit. Dit zorgt echter voor \1 probleem: Hoe zorgt de GUI ervoor dat het weet wanneer het scherm opnieuw getekend moet worden? Dit vraagstuk is perfect op te lossen met het Observer / Observable pattern. Wanneer via de Facade gebruikersinput ervoor zorgt dat de presentatie aangepast wordt (volgende slide, volgende item, open andere presentatie, etc), zal de presentatie een seintje geven aan de GUI. Dit is mogelijk door het interface Observer, welke een update methode beschikbaar stelt. De update methode uit Observer wordt ge\"{i}mplementeerd door de GUI, welke een repaint zal veroorzaken. Dit werkt in combinatie met de abstract class Observable, welke een manier kent om Observers te registreren en om Observers te seinen dat er wat veranderd is. Deze abstract class wordt gebruikt door de PresentationController. Waarom PresentationController deze abstract class gebruikt en niet Presentation zelf, wordt uitgelegd in het kopje \ref{sub:presCntrllr}.

\subsection{Factories}
Er is gekozen voor een strakke scheiding tussen creatie en gebruik van objecten. Om dit te verwezenlijken, zijn er meerdere Factories gemaakt. Deze Factories zijn allemaal terug te vinden in het bijgeleverde diagram.
\\\\
E\'{e}n van de Factories vergt wat uitleg. De ItemStylePool is een speciale soort Factory die slechts een paar instanties van het object ItemStyle maakt. In JabberPoint (uitgangspunt en uiteindelijke versie) staan de stijlen van de verschillende levels namelijk vast. Dit betekent dat er slechts vijf instanties van de class ItemStyle nodig zijn. Om te zorgen dat er geen nodeloze kopie\"{e}n van hetzelfde object zijn, is deze ItemStylePool ontworpen. De ItemStylePool heeft vijf instanties van ItemStyle en geeft een referentie naar \1 van deze vijf objecten wanneer daar om gevraagd wordt.

\subsection{Strategies}
\subsubsection{ItemParser}
\subsubsection{ItemEncoder}

\subsection{Draw functionaliteit}
In een eerdere iteratie van het ontwerp hebben we geprobeerd de GUI en de functionaliteit zo goed mogelijk te scheiden. Dat is aardig gelukt, op \1 vlak na: het tekenen van de items op het scherm. In een nieuwe iteratie van het ontwerp hebben we dit zo goed mogelijk proberen op te lossen. We hebben hier een soort Strategy pattern van proberen te maken, welke dynamisch bij elke aanroep van draw meegegeven wordt. De Model objecten zijn dus niet rechtstreeks afhankelijk van de juiste Strategy, maar deze wordt meegegeven bij de aanroep van drawUI. Op deze manier kan het Model de draw functie van de Strategy aanroepen. Op dit moment is de applicatie volledig gebouwd op Java Swing, maar dankzij deze speciale Strategy is het absoluut niet ondenkbaar dat de applicatie op ten duur ook bijvoorbeeld een GUI gebouwd op HTML kan ondersteunen, zonder daarvoor ontzettend veel aanpassingen voor te hoeven doen. Een nieuwe concrete implementatie van de DrawStrategy is dan namelijk voldoende.

\subsection{PresentationController}
\label{sub:presCntrllr}
In een eerdere iteratie van het ontwerp was gekozen om de abstracte class Observable te laten implementeren door Presentation. De Facade stuurde de commando's rechtstreeks door naar Presentation (waar toepasselijk) en Presentation gaf vervolgens een seintje naar de GUI dat het aangepast was. Dit heeft echter \1 groot probleem, namelijk dat wanneer er een nieuwe Presentation geopend werd, de GUI niet zomaar kon weten dat de oude Presentation niet meer relevant was. De GUI werd dan niet direct ge\"{u}pdatet bij het wijzigen van de te tonen Presentation, zoals na het laden van een ander bestand. Een ander gevaar bij het Observer / Observable pattern is dat oude (niet relevante) Presentation objecten niet automatisch meer opgeschoond worden door de Garbage Collector. De Observer heeft namelijk nog een referentie (via het aanmelden) naar de Presentation. Om deze redenen is gekozen om een PresentationController tussen de Facade en Presentation te zetten. Deze Controller vangt de opdrachten van de Facade op en speelt ze door naar Presentation. Deze Controller implementeert de abstrace class Observable in plaats van Presentation. Wanneer er een andere Presentation geladen wordt, is de Controller nog hetzelfde. De GUI kan dus nog steeds seintjes ontvangen van de Controller, ook al is de Presentation een hele andere instantie.

\section{Sourcecode}
\label{sec:source}
In deze sectie wordt de sourcecode beschreven. Allereerst zal er wat verteld worden over het refactorproces en wat pijnpuntjes waar we tegenaan liepen, waarna de applicatie zelf beschreven wordt.
\\\\
We zijn begonnen met een leeg project. Hier hebben we alle classes toegevoegd die de structuur van het ontwerp volgen. Connecties zijn tussen de classes gelegd en de Factories direct gemaakt. De code van de Factories is namelijk vrij triviaal. Verder is de GUI uit het vorige project genomen en gerefactored om in de nieuwe structuur te passen. De Facade is gemaakt met stubs (printjes om de werking te testen) en vervolgens de structuur van het Model. Daarna is het afhandelen van de bestanden gebouwd en het programma getest. Er werd op dat moment nog niets getekend op het scherm, maar de structuur werd perfect ingelezen.

Bij het bouwen van het draw mechanisme bleek al gauw dat we daar nog niet goed genoeg over nagedacht hadden. Het uitgangspunt was om View en Model te scheiden, maar om te tekenen bleek het Model de View nodig te zijn. Dit resulteerde in code die bijna net zo rommelig was als de oude code. Na overleg tussen Gerralt en Jan Jaap, is gekozen voor een soort Strategy structuur die meegegeven wordt in de parameters van de draw methode. Deze Strategy wordt gebruikt om het tekenen te abstraheren en nu is View en Model keurig losgekoppeld.
\\\\
Het opstarten van het programma gebeurt door het aanroepen van de main methode in JabberPoint. De class JabberPoint doet niets anders dan een Swing JFrame (JabberPointFrame) aanmaken, welke zichzelf verder opbouwt. Verder kijkt JabberPoint of er argumenten zijn meegegeven tijdens het opstarten. Wanneer er argumenten zijn meegegeven, zal JabberPoint de Facade zeggen om de presentatie te openen met als input het argument. Als er geen argumenten zijn, zal JabberPoint de standaard (demo) presentatie tonen. 

Tijdens het opstarten wordt ook de connectie tussen de GUI en het functionele deel opgezet aan de hand van het beschreven Observer / Observable pattern. Omdat er geen meerdere instanties zijn van de Observer en Observable, worden seintjes altijd maar keurig \1 keer verstuurd.
\\\\
De sourcecode is verder compleet van JavaDoc voorzien. De uitleg van specifieke items laten we dus graag aan deze JavaDoc over om te voorkomen dat we dingen dubbel beschrijven. Er is echter \1 constructie die nadere uitleg vergt en dat is de recursieve structuur van SlideItems. De oude structuur maakte het lastiger om de functionaliteit te introduceren zoals beschreven in de Feature request. Daar moet namelijk onderscheid zijn in level 1 items en onderliggende items, aangezien er door level 1 items (en de kinderen van dat level 1 item) gebladerd moet kunnen worden. Om het ontwerp flexibel te maken, is gekozen om de hele structuur op te bouwen door ook de level 2 items zijn "kinderen" te geven en niet alles onder level 1 items te hangen. Dit zorgt voor een recursieve opbouw, welke erg overeenkomt met hoe XML en JSON normaal gesproken opgebouwd is. Dit zorgt ervoor dat de code flexibeler is wanneer er meerdere bestandsformaten ondersteund moeten worden en wanneer er misschien later ook nog door level 2 items gebladerd moet kunnen worden bijvoorbeeld.

Maar, hoe werkt de structuur? Het opbouwen gebeurt in twee fasen, welke hier worden beschreven. De eerste fase is een SlideItem toevoegen aan de Slide. Als de Slide nog leeg is of als het een level 1 SlideItem betreft, wordt deze aan de Slide toegevoegd. Mocht dit niet gebeuren, komt de betreffende SlideItem in fase 2. In fase 2 zal de Slide zijn laatst toegevoegde SlideItem uit de lijst vertellen dat deze het toe te voegen SlideItem moet verwerken als \1 van zijn kinderen. Dit gebeurt met de aanroep \code{lastAdded.addChild(slideItem);}. Wanneer dit gebeurt, zal de \code{lastAdded} SlideItem gaan kijken of de toe te voegen SlideItem een sibling van zijn child (zelfde level als laatst toegevoegde child) of een child (lager level dan laatst toegevoegde child) betreft. Als het een sibling is, voegt de \code{lastAdded} SlideItem de toe te voegen SlideItem toe aan zijn eigen lijst (children). Mocht het een child van een child zijn, zal fase 2 zich herhalen voor het laatst toegevoegde child.

\end{document}
