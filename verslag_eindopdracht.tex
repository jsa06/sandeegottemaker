\documentclass[a4paper]{article}

\usepackage{xcolor}
\usepackage{fancyheadings}
\usepackage{booktabs}
\usepackage{enumitem}

\newcommand{\todo}[1]{\textcolor{red}{[#1]}}
\newcommand{\1}[0]{\'{e}\'{e}n}

\renewcommand\labelitemi{-}
\renewcommand{\figurename}{Figuur}
\renewcommand{\tablename}{Tabel}

\lhead{Open Universiteit}
\chead{IM0102, Design patterns}
\rhead{Eindopdracht}

\begin{document}
\pagestyle{fancy}

\section*{Studentgegevens}
\begin{description}
	\item [Cursuscode] IM0102
	\item JabberPoint
    \item Jan Jaap Sandee --- 852025385
	\item Gerralt Gottemaker --- 852083852
\end{description}
\todo{Verslag op taal- en spellingsfouten controleren}

\section*{Aanpak}
Om de opdracht succesvol af te ronden, is het zaak om goed samen te werken. We hebben het geluk collega's te zijn, wat het communiceren een aanzienlijk stuk gemakkelijker maakt. In dit hoofdstuk staat de aanpak beschreven die we gekozen hebben. De volgorde waarin het opgeschreven is, is ook de volgorde waarin we gewerkt hebben. Verder staat per item beschreven wie hieraan gewerkt heeft. Als niet vermeld wordt wie eraan gewerkt heeft, hebben beide teamgenoten eraan gewerkt.

Er is begonnen met het opzetten van een Git repository waarin Jan Jaap Gerralt en beide examinatoren heeft toegevoegd. Hier is vervolgens de beginsituatie van JabberPoint toegevoegd. Even later is hier ook een begin van het verslag en een TODO lijst aan toegevoegd. Het verslag is toegevoegd omdat het een LaTeX bestand is, welke prima door versiebeheer beheerd kan worden. Verder heeft Gerralt een Google Drive folder gemaakt en Jan Jaap uitgenodigd. Hier zullen diagrammen (met Draw.io) en andere belangrijke bestanden die niet geschikt zijn voor Git versiebeheer worden bijgehouden. Als laatst heeft Gerralt een Project Board aangemaakt waar overzichtelijk de TODO's wat betreft code en documentatie bijgehouden kunnen worden.

Hierna heeft Jan Jaap een eenvoudig diagram gemaakt van de huidige situatie van JabberPoint. Dit geeft een duidelijker beeld van de pijnpunten en maakt de geplande refactoring overzichtelijker.

Vervolgens is er een Probleem Analyse gemaakt waarbij alleen is gekeken naar de
functionaliteit van het systeem. De zogenaamde Mankala aanpak staat hierbij
centraal. Gedurende het maken van de analyse is nog niet gekeken naar bestaande
code van JabberPoint, om zo vooroordelen uit te sluiten. Hier is bewust voor
gekozen omdat we het idee hebben dat het zo gemakkelijker is om te denken in
termen van het domein en niet direct te denken in klassenamen. Jan Jaap heeft
hiervan het opzetje gemaakt, welke vervolgens uitgebreid is door Gerralt. Hierna
is samen gewerkt aan de uiteindelijke versie van de probleemanalyse. Beiden
kwamen tot vergelijkbare conclusies wat betreft de Probleem Analyse.

Aan de hand van de probleemanalyse is een klassendiagram opgebouwd. Hierbij zijn
de principes die genoemd worden in het tekstboek gevolgd, waarbij erg gelet is
op het principe dat patterns de context creëren van onderliggende patterns. De
design principes zoals high cohesion en low coupling staan hierbij natuurlijk
ook centraal. In het verslag zijn vervolgens de verschillende (belangrijke)
classes en structuren toegelicht.

Het maken van de klassendiagram is gedaan in persoon met een diagram op een
grootscherm en veel overleg over het ontwerp. hierin is in fasen gewerkt om tot
een ontwerp te komen. Beginnend bij de Gui en de Facade, hier kwam vervolgens de
persentatie zelf bij. Als laatste hebben we gekeken naar het afhandelen van
bestanden.

\\\\
\todo{Aanpak keuzen verantwoorden}
\\
\todo{Aanpak implementatie verantwoorden}


\section{Probleemanalyse}
Om JabberPoint goed te kunnen refactoren, is een probleemanalyse van de casus gemaakt. Deze probleemanalyse is op basis van de Mankala aanpak gedaan. Deze aanpak bestaat uit een aantal stappen, welke een steeds duidelijker beeld van het domein moeten opleveren. In de volgende koppen staan deze stappen uitgewerkt.

\subsection{Dingen}
We beginnen met het vastleggen van de entiteiten van het domein. Deze entiteiten, of "dingen", worden opgesomd in Tabel \ref{table:dingen}. De entiteiten die daar genoemd worden, zijn de entiteiten die van belang zijn voor het domein. Entiteiten die te maken hebben met de gebruikersinterface worden hier niet genoemd.
\begin{table}[!h]
\centering
	\begin{tabular}{lcl}
	\toprule
 	Concept & Sub-concept & Uitleg \\ \midrule
 	JabberPoint & - & Het programma. Kan presentaties afspelen.\\
 	Presentatie & - & Een enkele presentatie. Bevat slides.\\
 	Slide & - & Een pagina. Bevat elementen en een titel.\\
 	Element & Tekst & Bevat tekst voor op een slide.\\
 	 & Afbeelding & Bevat een afbeelding voor een slide.\\
 	Level & - & Het level van een element bepaalt wie de ouder is.\\
 	Stijl & - & Bevat info over het uiterlijk van een element.\\
 	Bestand & XML & Formaat waarin presentaties opgeslagen worden.\\
 	\bottomrule
	\end{tabular}
\caption{Lijst van entiteiten (dingen) binnen het domein.}
\label{table:dingen}
\end{table}
\\
Op het moment zijn er twee typen elementen: elementen die tekst bevatten en elementen die een afbeelding bevatten. Om het systeem zo flexibel mogelijk te maken, zal in het ontwerp rekening gehouden moeten worden met elementen met andere typen. Zo is het bijvoorbeeld denkbaar dat er in de toekomst een video element toegevoegd moet worden of iets dergelijks.
\\\\
Verder is het belangrijk om andere bestandsformaten alvast te ondersteunen in het ontwerp. Hierdoor behoudt het systeem zijn flexibiliteit. Op het moment wordt alleen XML ondersteund voor zowel het wegschrijven als het lezen. Echter zal in het ontwerp dus rekening gehouden worden met later toe te voegen bestandsformaten zoals JSON. Deze bestandsformaten zullen niet toegevoegd worden in zowel ontwerp als implementatie, echter zullen we bij de keuzen toelichten hoe we wel rekening houden met deze formaten.

\subsection{Acties}
Om de acties binnen het domein te kunnen gebruiken voor de analyse en het klassendiagram, is bepaalde informatie nodig. Om deze informatie te vergaren, wordt per actie de volgende vragen beantwoord:
\begin{itemize}[noitemsep]
\item Wie of wat heeft het initiatief om de actie te starten?
\item Aan welke regels moet voldaan worden voordat de actie uitgervoerd kan worden?
\item Wie of wat is gerelateerd aan de actie?
\item Is er aanvullende informatie nodig om de actie te kunnen uitvoeren?
\end{itemize}

In principe geldt voor elke actie dat het initiatief in de handen van de gebruiker ligt. Deze zal elke actie starten middels de grafische user interface (GUI). Echter zullen we bij het beantwoorden van de vraag ervan uitgaan dat de gebruiker de betreffende actie al begonnen is. Het initiatief ligt daarom bij de klasse / subroutine binnen het softwaresysteem en deze wordt dan ook genoemd. Om dezelfde reden wordt ook de GUI niet bij elk punt genoemd. Het moge echter duidelijk zijn dat de GUI voor de aansturing van het functionele deel van JabberPoint zorgt.

Deze sectie is vrij uitgebreid met als simpele reden dat deze sectie in principe de gehele functionele werking van JabberPoint toelicht. Hierdoor hoeft er in hoofdstuk \ref{sec:source} alleen aandacht besteed te worden aan de technische werking van het programma.

\subsubsection{Programma starten}
Wanneer het programma wordt gestart, zal JabberPoint opgestart worden met bepaalde standaard waarden. Bovenop JabberPoint zal een grafische user interface (GUI) gebouwd worden. Hoe de GUI communiceert met JabberPoint, is te vinden in hoofdstuk \ref{sec:ontwerp}. Om deze actie uit te voeren, is het aan te raden om een bestandsnaam mee te geven in de parameters. Wanneer dit niet gebeurt, zal het systeem opstarten met een standaard presentatie. Als aanvullende informatie is het goed om te weten dat de GUI hier los gezien moet worden van JabberPoint. JabberPoint is dus het functionele deel van het programma.

\subsubsection{Presentatie openen}
\label{subsub:presOpenen}
De tweede actie is het openen van een presentatie. De actie wordt gestart door het afhandelen van een gebeurtenis uit het menu en zal door moeten druppelen naar het functionele deel van JabberPoint. De actie wordt gestart door het afhandelen van een gebeurtenis uit het menu en zal door moeten druppelen naar het functionele deel van JabberPoint. JabberPoint en de Presentatie zijn gerelateerd aan deze actie. Verder moet genoemd worden dat momenteel geen bestand van het systeem gekozen kan worden. JabberPoint zal dus een standaard bestand (test.xml) openen. In het ontwerp zal wel rekening gehouden worden met een mogelijke uitbreiding om meerdere bestandsformaten te ondersteunen.

\subsubsection{Nieuwe presentatie starten}
De volgende actie is het starten van een nieuwe presentatie. Deze presentatie is volledig leeg. Wat het starten van de actie en de gerelateerde entiteiten betreft, geldt hetzelfde als bij de actie Presentatie openen. Het is verder goed om te noemen dat het starten van een nieuwe presentatie relatief weinig nut heeft. Er is immers geen functionaliteit om de presentatie aan te passen via de GUI.

\subsubsection{Presentatie opslaan}
Verder kunnen presentaties worden opgeslagen. Het gaat dan om de presentatie die op dat moment getoond wordt door de GUI. Deze actie wordt wederom gestart door het menu en ook hier geldt dat de actie door moet druppelen naar het functionele deel. Het opslaan lukt alleen als er een presentatie geopend is. Gelukkig is er altijd een presentatie geopend, dus dit mag geen problemen opleveren. Ook hier geldt dat JabberPoint, de Presentatie en het bestandsformaat gerelateerd zijn. Verder moet worden opgemerkt dat er geen feedback gegeven wordt wanneer het opslaan lukt. Dit is echter puur functioneel. Het zal dus niet in het ontwerp terug komen, maar wel in de implementatie. Ook hier wordt in het ontwerp rekening gehouden met eventuele later toe te voegen bestandsformaten anders dan XML.

\subsubsection{Slide weergeven}
\label{subsub:slideTonen}
Wanneer het programma wordt gestart, wordt een presentatie geopend (zie kopje \ref{subsub:presOpenen}). Wanneer dit gebeurt, wordt de eerste slide van deze presentatie weergegeven op het scherm. In eerste instantie wordt alleen de titel van deze slide weergegeven. Hoe de overige items op het scherm moeten komen, wordt geschreven in een aantal van de volgende kopjes. Deze actie is dus gerelateerd van de actie welke het openen van een presentatie beschrijft. Ook JabberPoint en de Presentatie zijn gerelateerd. Verder gelden er enkele regels waar eerst aan voldaan moet worden voordat deze actie uitgevoerd kan worden. Deze regels worden in paragraaf \ref{sub:regels} genoemd.

\subsubsection{Volgende item van slide weergeven}
\label{subsub:volgendeItem}
Dit is een van de acties die voortkomt uit de Feature request. Wanneer er op het pijltje naar beneden gedrukt wordt, zal het volgende item worden weergegeven. Hoe het item wordt weergegeven, hangt volledig af van de Stijl van een item. Alleen level 1 items zullen deze actie ondersteunen. Alle onderliggende items van level 1 items worden tegelijkertijd getekend op het moment dat deze actie uitgevoerd wordt. Wanneer deze actie vervolgens nog een keer uitgevoerd wordt, wordt het volgende level 1 item plus onderliggende items getekend. Gerelateerd aan deze actie zijn de entiteiten JabberPoint, de Presentatie, Slide, item (Element) en Stijl. Verder is de actie uit kopje \ref{subsub:volgendeSlide} gerelateerd.

\subsubsection{Laatste item van slide verbergen}
\label{subsub:vorigeItem}
Ook deze actie komt voort uit de Feature request en hangt nauw samen met de vorige actie. Wanneer er op het pijltje naar boven wordt gedrukt, wordt het laatst verschenen level 1 item weer verborgen. Ook hier geldt dat alle onderliggende items (de kinderen van het betreffende level 1 item) worden verborgen. Gerelateerd aan deze actie zijn JabberPoint, de Presentatie, Slide en item (Element). Ook hier geldt dat er een andere actie gerelateerd is, namelijk de actie uit kopje \ref{subsub:vorigeSlide}.

\subsubsection{Alle items van slide in \1 keer tonen}
\label{subsub:alleItems}
Het is ook mogelijk om alle items van een bepaalde slide in \1 keer te tonen. Wederom gaat het hier om een actie die voortkomt uit de Feature request. Wanneer op het pijltje naar rechts gedrukt wordt, worden alle nog verborgen items van een slide allemaal weergegeven. In principe is dit niets anders dan een aantal keer de actie van kopje \ref{subsub:volgendeItem} aanroepen. Daarom geldt dat alle gerelateerde regels en entiteiten hetzelfde zijn.

\subsubsection{Alle items van slide in \1 keer verbergen}
De laatste actie die uit de Feature request komt is het in \1 keer verbergen van alle op dat moment zichtbare items. Deze actie wordt uitgevoerd wanneer er op het pijltje naar links gedrukt wordt. Net als bij de vorige actie, geldt hier dat deze actie niets anders is dan een aantal keer een bepaalde actie uitvoeren, namelijk de actie van kopje \ref{subsub:vorigeItem}. Ook hier geldt daarom dat alle gerelateerde regels en entiteiten hetzelfde zijn.

\subsubsection{Volgende slide}
\label{subsub:volgendeSlide}
Naar de volgende slide gaan is de volgende actie. Deze actie zal gestart worden wanneer het volgende item van de slide weergegeven moet worden, maar deze niet meer beschikbaar is. Wanneer dit gebeurt, zal er gekeken worden of er nog een slide beschikbaar is. Mocht dit het geval zijn, zal deze worden getoond zoals beschreven in het kopje \ref{subsub:slideTonen}. Vervolgens kan de actie uit het kopje \ref{subsub:volgendeItem} gebruikt worden om de items te tonen of de actie uit kopje \ref{subsub:alleItems} om alle items in \1 keer te tonen. Deze actie moet dus, voordat het uitgevoerd kan worden, aan bovenstaande regels voldoen. De bovengenoemde acties zijn dus gerelateerd aan deze actie. Ook de GUI en de entiteiten JabberPoint en de Presentatie zijn gerelateerd.

\subsubsection{Vorige slide}
\label{subsub:vorigeSlide}
Naast naar de volgende slide gaan is de tegenovergestelde actie ook mogelijk. Wanneer de actie uit kopje \ref{subsub:vorigeItem} (vorige item) uitgevoerd wordt zonder dat er nog items op het scherm zichtbaar zijn, wordt er naar de vorige slide genavigeerd. Hierbij moet aan de regel voldaan worden dat er een vorige slide beschikbaar is. Mocht er niet aan deze regel voldaan worden, zal de actie simpelweg niet uitgevoerd worden. De genoemde regel is gerelateerd aan deze actie, evenals de genoemde actie. Verder zijn de entiteiten JabberPoint, Presentatie en Slide gerelateerd.

\subsubsection{Naar specifieke slide}
Ook is het mogelijk om rechtstreeks naar een specifieke slide te navigeren. De GUI zal vragen om een getal, welke vervolgens gecontroleerd wordt. Als blijkt dat de gebruiker een ongeldig getal heeft ingevuld (ongeldig betekent lager dan 1 of hoger dan het aantal slides), wordt de actie niet uitgevoerd. Mocht het een geldig getal betreffen, zal het systeem de juiste slide tonen zoals beschreven in het kopje \ref{subsub:slideTonen}. Deze actie is dan ook gerelateerd. Verder zijn de entiteiten JabberPoint, Presentatie en Slide gerelateerd, evenals de bovengenoemde regel.

\subsubsection{Informatie "Over / about" weergeven}
Als laatst is het mogelijk om bepaalde auteurs- en versieinformatie te tonen. Dit zal gedaan worden via het menu van de GUI. De GUI zal vragen aan JabberPoint wat de te tonen informatie is en deze tonen. Daarom is alleen de entiteit JabberPoint gerelateerd. Er is verder geen informatie nodig om deze actie uit te voeren. 

\subsection{Regels}
\label{sub:regels}

Er zijn regels waar bepaalde acties aan moeten voldoen voordat de betreffende actie uitgevoerd mag worden. Ook zijn er regels waar het hele systeem aan moet voldoen om succesvol uitgevoerd te kunnen worden. Verder zijn er voor deze opdracht ook relevante regels waar de gebruikersinterface mee te maken krijgt. Al deze regels zijn opgesomd in Tabel \ref{table:regels}. Elke regel bevat ook een korte beschrijving.

\begin{table}[!h]
\centering
	\begin{tabular}{ll}
	\toprule
 	Regel & Omschrijving \\ \midrule
 	Presentatie openen & Wanneer de applicatie gestart wordt met bestandsnaam, \\& wordt deze presentatie geopend. \\
 	 Bestandsformaat & In de huidige vorm van JabberPoint, wordt alleen XML \\& Ondersteund als formaat. Mogelijke Feature request.\\
 	Demo presentatie & Wanneer de applicatie gestart wordt zonder bestandsnaam, \\& wordt er een demo presentatie geopend.\\
 	Huidige presentatie & Er is ten alle tijden precies \1 presentatie open.\\
 	Minstens \1 slide & Om een presentatie weer te kunnen geven, moet er ten \\& minste \1 slide aanwezig zijn in de presentatie.\\
 	Level van items & Level 1 items zijn de root items. Hogere levels zijn \\& onderliggende items van level 1 items.\\
 	Styling van items & De styling van items moet per item kunnen verschillen.\\
 	Soort styling & Onder styling valt: Kleur, font, lettergrootte, indentatie \\& en lead. Lead is de afstand tot de voorgaande regel.\\
 	Vaste styling & Er is een voorgedefinieerde set styling beschikbaar.\\
 	Type van items & Items moeten van type kunnen verschillen. Het hangt \\& per type af of er wat gedaan wordt met de styling.\\
 	Beginstaat slide & Wanneer een slide weergegeven wordt, is initieel alleen \\& de titel zichtbaar. \\
 	Volgende slide & De volgende slide is alleen beschikbaar wanneer alle items \\& zichtbaar zijn en er een volgende slide beschikbaar is.\\
 	Vorige slide & De vorige slide is alleen beschikbaar wanneer alleen de \\& titel zichtbaar is en er een vorige slide beschikbaar is.\\
 	Specifieke slide & Er kan alleen naar een specifieke slide genavigeerd worden \\& wanneer er een geldige slidenummer is opgegeven.\\
 	Level bepalend & Het level van een element bepaalt wie de ouder van dat \\& element is. Hoe hoger het level, hoe dieper de structuur.\\
 	Edit niet mogelijk & In de huidige vorm van JabberPoint, is het editen van \\& slides niet mogelijk. Mogelijke Feature request.\\
  	\bottomrule
	\end{tabular}
\caption{Lijst van regels binnen het domein.}
\label{table:regels}
\end{table}

\subsection{Gewenste Update}
\todo{Kopje verder uitwerken, lopende zinnen maken}

Er is ook een verzoek voor het aanpassen van gedrag in de applicatie. Deze geeft
de volgende aanpassingen in de Probleemanalyse:

Acties: De gebruiker kan per element naar voren gaan in de huidige slide.

Regels: Wanneer alle elementen getoond worden van een slide, dan zal de volgende
'element' naar de volgende slide gaan.

De update voegt geen nieuwe dingen toe aan het ontwerp.

\section{Ontwerp}
\label{sec:ontwerp}
Het ontwerp is op te delen in vijf expliciete packages die elk een eigen
verantwoordelijkheid hebben. JabberPoint de main class instantieerd alleen de
SlideViewerFrame.

\subsection{GUI}
De GUI is alleen veranwoordelijk voor de weergave aan de gebruiker.
SlideViewerFrame bouwt de gehele interface op. De Gui heeft geen idee van de
huidige presentaties, maar vertrouwd op een observer om te weten wanneer er iets
nieuws getoond moet worden.

\subsection{Controllers}
Belangrijkste controller is de JabberPointFacade, deze behandelt de verder
commands naar de rest van het programma voor de GUI. JabberPointFacade is ook
een Singleton class gezien er ten alle tijde maar 1 instantie van hoeft te
zijn.

De Facade registreert ook de Observer voor SlideViewerComponent. Dus de
koppeling tussen Presentation en SlideViewerComponent bestaat alleen maar via de
observer pattern.

\subsection{Models}
Models houden zich alleen bezig met het ordenen van de data. Belangijkste
hierarchie is Presentation -> Slide -> SlideItem. In een presentatie kunnen
SlideItems een level hebben. Dit level wordt bepaald door de mogelijkheid voor
SlideItems om zelf onderliggende SlideItems te bevatten.

Styles zijn een vast gegeven en worsen in een ObjectPool bewaard en bij creatie
van slideitems toegewezen.

Presentation is een observable, wanneer deze wijzigt van slide dan wordt de
SlideViewerComponent op de hoogte gezet dat er nieuwe inhoud is.

\subsection{Factories}
De Factories zijn verantwoordelijk voor het aanmaken van alle objecten. De
FileParserFactory en FileEncoderFactory zijn er zodat de juiste strategy wordt
toegepast op een file. Gezien de Strategy Pattern geen rekening houd met
objectcreatie.

De Factories voor de Presentaties zijn er om creatie te scheiden. Er was een
argument geweest voor Presentaties die zijn eigen slides aanmaakt, en slides
die zijn eigen Items aanmaakt, echter dit zou betekenen dat de Presentatie
rechtstreeks toegang moet hebben tot de filedata wat niet wenselijk is.

Verder is er een ObjectPool voor de Styles. Deze pattern wordt gebruikt om te
waarborgen dat er maar vijf stijlen beschikbaar zijn. Anders is er het risico
dat er voor elke nieuwe slideitem er een nieuw object wordt aangemaakt.

\subsection{FileHandlers}
De Filehandler is op zijn beurt weer een soort Facade voor de JabberFacade.
Namelijk JabberFacade heeft geen weet van hoe presentaties worden uitgelezen of
opgeslagen. Deze verteld gewoon aan de FileHandler wat er moet gebeuren.

Hierbij is er een concrete splitsing gedaan tussen het uitlezen en het opslaan.
Daarmee kunnen apparte parsers en encoders, dit omdat deze taken asymetrisch
zijn.

\section{Keuzen}
\todo{Gemaakte keuzen verantwoorden op twee gebieden: Ontwerp en Implementatie}
\subsection{Facade}
De Facase was ons uitgangspunt bij het ontwerpen van deze tool. Alle acties
waren gebruiker gestuurd welke op twee manieren aangestuurd konden worden. Via
een menubalk of via toetsenbordcommando's. Hierdoor werdt het al snel duidelijk
dat deze commando's op een centraal punt gebracht moesten worden voor de verder
aansturing van de applicatie.

\subsection{Observer}
De GUI weet eigenlijk niks van de werking van het programma. De GUI is eigenlijk
ook alleen maar geinteresseerd in weten wat er weergegeven moet worden. Zodra
een presentatie van slide wisselt dan geeft deze aan dat er iets is gewijzigt in
de weergave. De GUI gaat dan kijken wat deze wijziging is.

\subsection{Factories}
Er is gekozen voor een strakke scheiding tussen creatie en gebruik van objecten.

\subsection{Strategy}




\section{Sourcecode}
\label{sec:source}
\todo{Sourcecode nader beschrijven}
\\
\todo{Proces van refactoring beschrijven}
\\
\todo{Proberen waar we tegen aan liepen beschrijven}
Koppeling GUI en Presentatie gescheiden houden. Zorgen dat Presentatie niet te
veel van grafische dingen afweet.

Het aanmaken van een nieuwe presentatie binnen het gekozen ontwerp houden.
Dit hebben we opgelost door een EmptyFileParser aan te maken.



\end{document}
