\documentclass[a4paper]{article}

\usepackage{xcolor}
\usepackage{fancyheadings}


\newcommand{\todo}[1]{\textcolor{red}{[#1]}}
\lhead{Open Universiteit}
\chead{IM0102, Design patterns}
\rhead{Eindopdracht}

\begin{document}
\pagestyle{fancy}

\section*{Studentgegevens}
\begin{description}
	\item [Cursuscode] IM0102
	\item Jabberpoint
    \item Jan Jaap Sandee
	\item 852025385
	\item Gerralt Gottemaker
	\item [Studentnummer] \todo{invullen}
\end{description}

\section*{Aanpak}
\todo{<Geef aan hoe jullie de opdracht hebben aangepakt en wie wat heeft gedaan, maximaal 1 A-4. Geef expliciet aandacht aan de volgorde van activiteiten>}
Allereerst is er een eenvoudig diagram gemaakt van de huidige situatie. Dit ten
behoeve van geplande refactoring en aanpassingen.

Vervolgens is er een Probleem Analyse gemaakt waarbij alleen is gekeken naar de
functionaliteit van het systeem. Zonder te kijken naar de code die er al is.


\section{Probleemanalyse}
JabberPoint is een eenvoudige presentatie tool. Op basis van Dingen, Regels, en
Acties ga ik het programma uitwerken om tot een degelijk ontwerp te komen.
Hierin kijken we eerst naar de acties in het kader van wat een gebruiker met die
programma kan doen. Dan de dingen die hier voor worden gebruikt, en als laatste
de regels waar het aan moet voldoen.

\subsection{Acties}
De gebruiker kan een bestand openen met daarin een presentatie.

De gebruiker kan naar de volgende en vorige slides gaan.

Update: De gebruiker kan per element naar voren gaan in de huidige slide.

\subsection{Dingen}

Er is een Presentatie, deze heeft Slides, en elke Slide heeft een aantal
elementen welke kunnen bestaan uit text of een afbeelding.

Er is een menu, welke een aantal acties voor de gebruiker heeft.

\subsection{Regels}
De gebruiker kan niet terug navigeren bij de eerste slide.

De gebruiker kan niet vooruit navigeren bij de laatste slide.

Als er geen bestandsnaam is opgegeven wordt er een standaard bestand geopend.

Update: Wanneer alle elementen getoond worden van een slide, dan zal de volgende
'element' naar de volgende slide gaan.

Omgekeerd bij het terug gaan zal naar de vorige slide worden gegaan welke
volledig getoond wordt met alle elementen.

Wanneer gebruikt gemaakt wordt van navigatie om direct naar een vorige of
volgende slide te gaan, dan wordt de slide in being status getoond.

\section{Ontwerp}


\section{Keuzen}



\section{Sourcecode}

\end{document}
